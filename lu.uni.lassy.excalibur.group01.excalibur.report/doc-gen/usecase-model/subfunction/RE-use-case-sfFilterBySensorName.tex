\subsubsection{subfunction-sfFilterBySensorName}

\label{RE-use-case-sfFilterBySensorName}


The technician can filter the sensors by type , this allows him to check if there is a sensor which gives as out
put the wrong values. 		  


\begin{usecase}
  \addheading{Use-Case Description}
  \addsingletwocolumnrow{Name}{sfFilterBySensorName}
  \addsingletwocolumnrow{Scope}{system}
  \addsingletwocolumnrow{Level}{subfunction}
  

\addrowheading{Primary actor(s)}
\addnumberedsinglerow{}{\msrcode{actTechnician[active]}}



\addrowheading{Goal(s) description}
\addsinglerow{The technician can filter the sensors by type , this allows him to check if there is a sensor which gives as out
put the wrong values. }

\addrowheading{Protocol condition(s)}
\addnumberedsinglerow{}{
The system is booted up the technician needs to be on he's home screen and there needs to be at least 2 different sensor types. 
}

\addrowheading{Pre-condition(s)}
\addnumberedsinglerow{}{
none
}

\addrowheading{Main post-condition(s)}
\addnumberedsinglerow{}{
The sensor table gets filtered by sensor type selected.
}

\addrowheading{Additional Information}
\addsinglerow{
In case there is no sensor of a given selected type the displayed table will be empty since there are no sensors of the selected type.
}

\end{usecase} 

