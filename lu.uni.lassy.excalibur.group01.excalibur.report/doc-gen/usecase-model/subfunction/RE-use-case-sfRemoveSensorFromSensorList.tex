\subsubsection{subfunction-sfRemoveSensorFromSensorList}

\label{RE-use-case-sfRemoveSensorFromSensorList}


The technician has the ability to remove a sensor from the sensor table. The sensor can
 only be deleted by the technician. In addition the technician has no constraints by doing it t can be for any reason. 
One reason might be for the re-installation of the sensor to an other room.


\begin{usecase}
  \addheading{Use-Case Description}
  \addsingletwocolumnrow{Name}{sfRemoveSensorFromSensorList}
  \addsingletwocolumnrow{Scope}{system}
  \addsingletwocolumnrow{Level}{subfunction}
  

\addrowheading{Primary actor(s)}
\addnumberedsinglerow{}{\msrcode{actTechnician[active]}}



\addrowheading{Goal(s) description}
\addsinglerow{The technician has the ability to remove a sensor from the sensor table. The sensor can
 only be deleted by the technician. In addition the technician has no constraints by doing it t can be for any reason. 
One reason might be for the re-installation of the sensor to an other room.
}

\addrowheading{Protocol condition(s)}
\addnumberedsinglerow{}{
The system is booted up and running the technician has to be logged in and be on he's home screen. There needs also to be
a sensor inside the sensor table.
}

\addrowheading{Pre-condition(s)}
\addnumberedsinglerow{}{
none
}

\addrowheading{Main post-condition(s)}
\addnumberedsinglerow{}{
The selected sensor is removed from the sensor table.
}

\addrowheading{Additional Information}
\addsinglerow{
In case a request for a sensor has been made for a new sensor which is broken then 
this will be the last step after adding the sensor .The after adding the new one he can delete the old one.
}

\end{usecase} 

