\chapter{Manager}
\label{chap:appendix_Manager}


\section{Settings}
\label{sec:appendix_Settings}
\mbox{} \par
\noindent\centerimg[width=\paperwidth]{images/appendix_images/Settings.eps}

The 'Settings' page can be reached on the 'Home' page. This page is reserved for
the Manager and can only then be accessed if the user is logged in as a manager.
Here you can manage the current Rooms and Security Camera inside the software.

\subsection{Room Input Formular (SE 1)}
The creation of a new room can be done inside this section. The Manager must
fill out all input fields to create a new room. If an input is invalid you will
receive a error message which can be lookep up in den Error section of the user
manual.

\subsection{Security Camera Input Formular (SE 2)}
The creation of a new security camera can be done inside this section. The
Manager must fill out all input fields to create a new room. If an input is
invalid you will receive a error message which can be lookep up in den Error
section of the user manual.

\subsection{Room List (SE 3)}
Here you see all current room inside the system. If you want to delete a room no
longer used, it can be deleted by using the red remove button inside the row of
the represented room.

\subsection{Security Camera List (SE 4)}
Here you see all current security cameras inside the system. If you want to
delete a security camera no longer used, it can be deleted by using the red
remove button inside the row of the represented room.


\newpage
\section{Security Camera}
\label{sec:appendix_SecurityCamera}
\mbox{} \par
\noindent\centerimg[width=\paperwidth]{images/appendix_images/SecurityCamera.eps}

The 'Security Camera' page can be reached on the 'Home' page. This page is
reserved for the Manager and can only then be accessed if the user is logged
in as a manager. Here you can look at all security cameras or a single security
camera.

\subsection{Security Camera View (CA 1)}
The grid represents the different cameras that can be shown if they are
connected.

\subsection{Camera Select List (CA 2)}
Select the camera you want to view and then use the button 'View Camera' (CA 3)

\subsection{View Camera Button (CA 3)}
The camera selected from the select list in (CA 2) will be shown in the Security
Camera View (CA 1)

\subsection{4-Camera View Button (CA 4)}
If a single camera is shown, you can use this button to show again 4 cameras at
the same time.



\newpage
\section{Crop Manager}
\label{sec:appendix_CropManager}
\mbox{} \par
\noindent\centerimg[width=\paperwidth]{images/appendix_images/CropManager.eps}

The 'Crop Manager' page can be reached on the 'Statistics' page inside the
Crops Statistics. This page is reserved for the Manager and can only then be
accessed if the user is logged in as a manager. Here you can manage all current
crops and create new crops.

\subsection{Crop Input Formular (CM 1)}
The creation of a new Crop can be done inside this section. The manager must
fill out all input fields to create a new crop inside the system. If an input is
not valid you will receive an error message which can be looked up in the Error
section of the user manual.





\section{Manager Home screen}
\label{sec:appendix_Manager_Home_Screen}

\begin{figure}[H]
\includegraphics[width=1\textwidth]{images/appendix_images/ManagerHomeRRR.eps}
\end{figure}

The Manager home screen is the screen which is dedicated for his tasks most of
his features are executed here like removing seed completly from the inventory
and adding seeds to the inventory.


\subsection{Inventory of Seeds (MH 5)}
The table with the title Inventory of Seeds,displays all the content of seeds
left in the global inventory.

\subsection{Gardener retrives from Inventory (MH 1)}
The table with the title Gardener retrives from inventory displays at which time
a given gardener has retrived a given amount of seeds.

\subsection{Delete from Inventory of seeds (MH 4)}
This red button executes a function/ future which allows to remove a selected
seed completly from the sees inventory table.

\subsection{Go to manager Request screeen (Special)}
This grey button can have 2 colors grey and red. By pressing the button the
manager gets redirected to the Manager request screen. In case the Button is red
the is a request that needs to be threated. If the button is Grey there is a
request which isn't urgent.In addition the access to this button is restricted
the button appears only if their is indeed a request.Else the functionalities
from the request screen are all disabled.

\subsection{Add seed to Inventory Of Seeds (MH 02)}
This orange button executes a function/feature which allows to add a seed to 
the Inventory of Seeds. The input formular (MH 03) has to be completed before
executing the feature .

\subsection{Input formular to add a seed/item MH 03}
This input fields are used to add the information to the inventory seed table. In addition if the input fields aren't valid an given error will be thrown.

\subsubsection{Errors 31}
Constrains are on the textfields. In case of a wrong input look up the Error
Section of the user manual.


















\section{Manager Request Screen}
\label{sec:appendix_Manager_Request_Screen}

\begin{figure}[H]
\includegraphics[width=1\textwidth]{images/appendix_images/ManagerRequestRRR.eps}
\end{figure}

The Manager Request screen is the screen which is dedicated to the request of an
Item from  Actors like the Gardener and the Technician or from the System. 


\subsection{Go to manager 1 screen (MR 01)}
This button redirects the manager to the Manager home screen.

\subsection{Ordering Seeds request by amount (MR 02)}
The manager can press the button arrow up or down for ascending or descening
order of the list.

\subsection{Accept requested seeds (MR 04 green)}
This green button allows the manager to execute a feature/function which accepts
a requested crop. 
\subsubsection{2 cases MR 04}
\tab{Request from the system automatically refills the the requested seeds.} 

\tab{Request from the Gardener the manager needs to add the requested crop by
it's own by adding a seed on the home screen.}

\subsection{Decline a seed Request (MR 04 red)}
This red button allows the manager to execute a function/feature which allows
him to decline a request for a seed.

\subsubsection{Errors 33}
Error can be thrown in case the request is 0.


\subsection{Decline a Sensor Request (MR 06 red)}
This red button allows the manager to execute a function/feature which allows
him to decline a request for a sensor.
\subsection{Accept a Sensor Request (MR 06 green)}
This green button executes a function/feature that allows to accept a request
for a sensor from the technican.




