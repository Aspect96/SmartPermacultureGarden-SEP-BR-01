
\chapter{Error messages and problem resolutions}
\label{chap:error_messages}

All known problems are listed here below with the given error codes and how to
solve them. In case an error is not listed here below please feel free to
contact us \hyperref[sec:Contact]{Click on me}.


\section{\textbf{\textcolor{red}{Error message 1}}}
\hrule
\vspace{0.5cm}
\textbf{Ph value is not correct (1)}
\subsection{Problem identification}
Ph value is defined on a scale of 1 to 14.  Where 1 is most acidic and 14 most
basic or alkaline.

\subsection{Probable cause}

The problem is due to the constraints on the Ph value text field that means the
value entred here has to be between 1 and 14.

\subsection{Corrective actions}
Enter a valid number between 1 and 14.
\vspace{0.5cm}
\hrule


\break

\section{\textbf{\textcolor{red}{Error message 2}}}
\hrule
\vspace{0.5cm}
\textbf{Crop Name not valid (2)}
\subsection{Problem identification}
The name text field input was not valid.

\subsection{Probable cause}
Due to the constraints on the Name textfield the error happens.

\subsection{Corrective actions}
The name has to be at least 5 caracters long.
\vspace{0.5cm}
\hrule
\hfill
\section{\textbf{\textcolor{red}{Error message 3}}}
\hrule
\vspace{0.5cm}
\textbf{Amount no valid input (3)}
\subsection{Problem identification}
The amount text field input was not valid.

\subsection{Probable cause}
Due to the constraints on the amount input the error happend.

\subsection{Corrective actions}
To solve the Error 3 you have to enter an amount which is higher than 0.
\vspace{0.5cm}
\hrule
\break

\section{\textbf{\textcolor{red}{Error message 4}}}
\hrule
\vspace{0.5cm}
\textbf{Temperature Prefernce is not valid (4)}

\subsection{Problem identification}
The entered text field input was not valid.

\subsection{Probable cause}
Due to the constraints on the Temperature Preference input the error happend.

\subsection{Corrective actions}
To solve the Error 4 you have to enter on the Temperature Preference a value
between 15 and 40 degrees.
\vspace{0.5cm}
\hrule
\hfill
\section{\textbf{\textcolor{red}{Error message 5}}}
\hrule
\vspace{0.5cm}
\textbf{Sensor Name not valid (5)}
\subsection{Problem identification}
The input of text field Name is not valid.

\subsection{Probable cause}
Due to the constraints on  input of Sensor Name text field the error happend.

\subsection{Corrective actions}
To solve the Error 5 you have to enter on the Sensor Name Preference a Number at
the end of the input. For good pratice use the Room number where you want to add
it as last index of the sensor name.
\vspace{0.5cm}
\hrule
\break

\section{\textbf{\textcolor{red}{Error message 6}}}
\hrule
\vspace{0.5cm}
\textbf{Room input text is not valid (6)}
\subsection{Problem identification}
The cause of the error occurs due the non valid input on the Room text field
input.

\subsection{Probable cause}
Due to the constraints on the on the input the error happens.

\subsection{Corrective actions}
To solve the Error 6 trouble you have to input on the text field Room a number
which is higher than 0.
\vspace{0.5cm}
\hrule
\hfill
\section{\textbf{\textcolor{red}{Error message 7}}}
\hrule
\vspace{0.5cm}
\textbf{Technician name is not valid (7)}

\subsection{Problem identification}
The cause of the error is due to an non valid input on the Name text field.

\subsection{Probable cause}
Due to the constraints on the input field name the error happens.

\subsection{Corrective actions}
To solve the Error 7 you have to input on the name text field a name which has
no numbers.

\vspace{0.5cm}
\hrule
\break

\section{\textbf{\textcolor{red}{Error message 8}}}
\hrule
\vspace{0.5cm}
\textbf{Reason is not valid (8)}
\subsection{Problem identification}
The cause of the error is due to a non valid input on the Reason text field.

\subsection{Probable cause}
Due to the constraints on the input field Reason the error happens.

\subsection{Corrective actions}
To solve the Error 8 you have to enter at least a 4 digit long input on the text
field Reason.

\vspace{0.5cm}
\hrule


\section{{\textbf{\textcolor{red}{Error message 9}}}} 
\hrule
\vspace{0.5cm}
\textbf{Deletion was declined (9)}

\subsection{Problem identification}
The manager tries to decline a request.

\subsection{Probable cause}
Due to the constrain on the button after pressing it.

\subsection{Corrective actions}
To solve the Error 9 you can not delet a request where the value is 0. 0 means
there is no request at the moment.

\vspace{0.5cm}
\hrule
\break

\section{{\textbf{\textcolor{red}{Error message 10}}}}
\hrule
\vspace{0.5cm}
\textbf{Retriving Failed (10)}

\subsection{Problem identification}
Retriving modify on the crops inventory failed the input amount is the issue.

\subsection{Probable cause}
Due to the constraint on the amount text field the error happens.

\subsection{Corrective actions}
To solve the Error 10 you need to input an amount which is less or equal the
amount left of crops for the given crop which you want to retrive.

\vspace{0.5cm}
\hrule






