\chapter{Introduction}
\label{chap:introduction}

\section{Scope}

This user manual will explain you how to use our software efficiently. It will
show you how the different sensors interact with each other and how statistics
are used to schedule tasks. It will also show you how you can properly interact
with the software. It will be useful reference material during the use of the
Software. This document will not show you how to integrate the software in your
specific environment. Thus not showing you how to create and implement databases
nor does it show you how to care for the different plants. This document wills
also not show you how to properly install the sensors.


\section{Purpose}

This document defines all essential information for the user to use the software application. 
The system functions are shown step by step for the user. In case of a
\textbf{\emph{\glspl{crisis}}} the user can look up in this file with the
given error code which action has to be taken accordingly.
In addition the definition of different icons will be shown and explained.





\section{Intended audience}

The software is intended to be primarily used in the enterprise sector more
specially the agriculture sector. The manual  will help the people concerned
with operating the software correctly. The gardener can use this document to
understand how he can interact with the system and retrieve information about
the inventory. The manager can learn how to add tasks to the schedule correctly,
how to access camera. The technician will know where to look for problems with
the sensors as well as what alerts he can expect. The gardener and the
technician can find information about the schedule and what task can be given to
the respecting personnel.

\section{\mysystemname}

The software is used in agriculture/gardening that means planting and harvesting
plants. The main purpose is to do so in the most efficient way by using
different hardware like sensors. Basically the software allows the user to check
if the soil needs nutrition or water to allow for a optimal growing of the
plants. Also the software gives tasks to the gardner so that no mistakes are
made by the individual human. The software creates statistics with the data
gather by the sensors about the plants and their environment. The software gives
insight to the results achieved with varying environmental conditions.


\subsection{Actors \& Functionalities}
Overview of all the \textbf{\emph{\glspl{actor}}} interacting with the software
being them either humans (called end-users in the standard
\cite{IEEE-2001-userdocumentation}) or not. For each actor, describe the main
software functions that are offered to him. Structure of this sub-section MUST
be by actor/functionalities.


\subsection{Operating environment}
Brief overview of the infrastructure on which the software is deployed and used.

\section{Document structure}  
Information on how this document is organised and it is expected to be
used. Recommendations on which members of the audience
should consult which sections of the document, and explanations about the used
notation (i.e. description of formats and conventions) must also be provided.

