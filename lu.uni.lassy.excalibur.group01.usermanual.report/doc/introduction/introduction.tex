\chapter{Introduction}
\label{chap:introduction}

\section{Scope}
This user manual will explain you how to use our software efficiently. It will
show you how the different sensors interact with each other and how statistics
are used to schedule tasks. It will also show you how you can properly interact
with the software. It will be useful reference material during the use of the
software. This document will not show you how to integrate the software in your specific environment.


\section{Purpose}
In this section you explain the purpose (i.e. aim, objectives) of the user's
manual. In the following some examples of opening statements to be used in this
section.

The purpose of this document is \ldots

This document defines \ldots

This document is meant to \ldots



\section{Intended audience}
Description of the categories of persons targeted by this document together with the description of how they are expected to exploit the content of the document.


\section{\mysystemname}
Brief overview of the software application domain and main purpose.
This is on purpose wrong. Its a testsss.


\subsection{Actors \& Functionalities}
Overview of all the \textbf{\emph{\glspl{actor}}} interacting with the software
being them either humans (called end-users in the standard
\cite{IEEE-2001-userdocumentation}) or not. For each actor, describe the main
software functions that are offered to him. Structure of this sub-section MUST
be by actor/functionalities.


\subsection{Operating environment}
Brief overview of the infrastructure on which the software is deployed and used.

\section{Document structure}  
Information on how this document is organised and it is expected to be
used. Recommendations on which members of the audience
should consult which sections of the document, and explanations about the used
notation (i.e. description of formats and conventions) must also be provided.





