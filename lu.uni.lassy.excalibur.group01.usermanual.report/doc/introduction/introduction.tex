\chapter{Introduction}
\label{chap:introduction}

\section{Scope}
 
This section has to provide the scope of the user's manual document.
In the following some opening statements to use when providing the
information corresponding to this section.

This document provides \ldots
%Example: This document provides minimum acceptable information for knowing how
% to use the software system \mysystemname.


This document does not \ldots 
 
This document is not \ldots
%Example: This document is not intended to provide information about how to
% connect, deploy, configure, or use any external device or
% third-party software system that is rqeuired for the correct funcitoning of
% \mysystemname.

 
This document may be used with \ldots
%This document may be used with other documents provided by third-party
% companies to have an overall view and correct understanding of the environment
% and procedures where the software system \mysystemname is aimed to be deployed
% and run.




\section{Purpose}
In this section you explain the purpose (i.e. aim, objectives) of the user's
manual. In the following some examples of opening statements to be used in this
section.

The purpose of this document is \ldots

This document defines \ldots

This document is meant to \ldots



\section{Intended audience}
Description of the categories of persons targeted by this document together with the description of how they are expected to exploit the content of the document.


\section{\mysystemname}
Brief overview of the software application domain and main purpose.
This is on purpose wrong. Its a testsss.


\subsection{Actors \& Functionalities}
Overview of all the \textbf{\emph{\glspl{actor}}} interacting with the software
being them either humans (called end-users in the standard
\cite{IEEE-2001-userdocumentation}) or not. For each actor, describe the main
software functions that are offered to him. Structure of this sub-section MUST
be by actor/functionalities.


\subsection{Operating environment}
Brief overview of the infrastructure on which the software is deployed and used.

\section{Document structure}  
Information on how this document is organised and it is expected to be
used. Recommendations on which members of the audience
should consult which sections of the document, and explanations about the used
notation (i.e. description of formats and conventions) must also be provided.





