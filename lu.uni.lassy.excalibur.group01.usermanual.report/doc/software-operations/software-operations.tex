\chapter{Software operations}
\label{chap:soptware_operations}


Explain each allowed software operations (i.e. an atomic unit of treatment, a service, a functionality) including a brief description of the operation, required parameters, optional parameters, default options, required steps to trigger the operation, assumptions upon request of the operation and expected results of executing such operation.
Describe how to recognise that the operation has successfully been executed or
abnormally terminated. The template given below (i.e. section \ref{operation:MyOperation} has to be used).

Group the operations devoted to the needs of specific actors. Common
operations to several actors may be grouped and presented once to avoid redundancy.


\section{MyOperation}
\label{operation:MyOperation}
The system operator creates and adds a new crisis to the system after being 
informed by a third party (citizen, organization) and selects a crisis handler for the crisis.

\begin{description}

\item \textbf{Parameters:} Reporter Personal Information, Crisis Information, Crisis Handler
\item \textbf{Precondition:} The system operator is logged in and has received information from a reporter.
\item \textbf{Post-condition:} A new crisis has been added to the system and the new crisis has been assigned to a crisis handler, the Handler has received an automatic notification from the system.
\item \textbf{Output messages:} The selected Crisis Handler will be notified
automatically once the crisis has been created.

\item \textbf{Triggering:}
\begin{enumerate}
\item From within the crisis management window fill out the required entries related to the personal information of the reporter such as name and phone number.
\item Fill out the entries related to the crisis type, impacted area, priority, description, GPS coordinates, address and finally choose a Crisis Handler from the combo box.
\item Click on the Submit button in and add the entry to the database.
\end{enumerate}

 
\end{description}

 
\subsection{MyExample1}
Examples should illustrate the use of \textbf{complex operations}.

Each example must show how the actor uses the software operation under
description to achieve (at least one of) its expected outcome.

It might be required to include GUI screenshots to illustrate the example.




\section{Adding task to gardeners schedule}
\label{operation:addTaskGardener}
The manager creates and adds a new task to be added to schedule.


\begin{description}

\item \textbf{Parameters:} Task name, Task description, Room, Gardeners name,
Name of the task
\item \textbf{Precondition:} The manager is logged in
\item \textbf{Post-condition:} A new task has been added to the schedule and the
new task has been assigned to the specified gardener.
\item \textbf{Output messages:} The manager will be notified that the task has
been created.

\item \textbf{Triggering:}
\begin{enumerate}
\item From within the Manager Schedule screen, the manger fills out the 
required entries related to the task information like the name of the task or
the date.
\item Click on the add button and add the task to the schedule.
\end{enumerate}

 
\end{description}

 
\subsection{MyExample1}
The manager wants to add a watering task to the schedule, so fills out the
textinputfields as shown in the image and then he clicks on the add button.




\section{Automatical request for crops}
\label{operation:RequestForCrops}
The system checks the crops if any specifique crop is less then 10 and requests
crops.

\begin{description}

\item \textbf{Parameters:} ListOfCrops
\item \textbf{Precondition:} The system is bootedup and any actor is logged in.
\item \textbf{Post-condition:} Request table from Manager Screen2 updated with
the amount of crops which have to be refilled.

\item \textbf{Triggering:}
\begin{enumerate}
\item System checks every singel amount left of crops in the vegtable table.
\item In case that an amount is less than 10.
\item System send request to the manager.
\end{enumerate}
\end{description}

\subsection{Example of Automatical Frood/Vegtable Request}
Gardener retrives a chosen amount of crops of a specific vegtable or frood.
If the amount of the specifique crop is less than 10.
The System will send the request to the manager request table.




\section{Build Temperature Line Diagram}
\label{operation:BuildTemperatureDiagram}
The system draws the line diagram for the temperature of the selected room of
the dropdown menu next to it.

\begin{description}

\item \textbf{Parameters:} HistoryOfTemperaturesInAllRooms
\item \textbf{Precondition:} The system is bootedup and any actor is logged in.
\item \textbf{Post-condition:} The corresponding diagram will be drawn and
shown.

\item \textbf{Triggering:}
\begin{enumerate}
\item If at any time the \emph{statistics} page is accessed.
\end{enumerate}
\end{description}

\subsection{Example of Build Temperature Line Diagram}
The logged in user wants to see the statistics of the past days.
The user clicks on the \emph{statistics} page in the menu.
The system will now draw the temperature line diagram.




\section{Build Current Stock Bar Diagram}
\label{operation:BuildCurrentStockDiagram}
The system draws the bar diagram of the current stock of the inventory.

\begin{description}

\item \textbf{Parameters:} CurrentStockOfSeeds
\item \textbf{Precondition:} The system is bootedup and any actor is logged in.
\item \textbf{Post-condition:} The corresponding diagram will be drawn and
shown.

\item \textbf{Triggering:}
\begin{enumerate}
\item If at any time the \emph{statistics} page is accessed.
\end{enumerate}
\end{description}

\subsection{Example of Build Current Stock Bar Diagram}
The logged in user wants to see the current stock of the inventory.
The user clicks on the \emph{statistics} page in the menu.
The system will now draw the current stock bar diagram.




\section{Build Water Consumption And Luminosity Level Bar And Line Diagram}
\label{operation:BuildWaterConsumptionAndLuminosityLevelDiagram}
The system draws the bar and line diagram for the water consumption in
milliliter and the luminosity level in spectral power of the selected room of
the dropdown menu next to it.

\begin{description}

\item \textbf{Parameters:} HistoryOfWaterConsumptionInAllRooms,
HistoryOfLuminosityLevelInAllRooms
\item \textbf{Precondition:} The system is bootedup and any actor is logged in.
\item \textbf{Post-condition:} The corresponding diagram will be drawn and
shown.

\item \textbf{Triggering:}
\begin{enumerate}
\item If at any time the \emph{statistics} page is accessed.
\end{enumerate}
\end{description}

\subsection{Example of Build Water Consumption And Luminosity Level Bar And Line
Diagram}
The logged in user wants to see the statistics of the past days.
The user clicks on the \emph{statistics} page in the menu.
The system will now draw the temperature line diagram.




\section{Show Crop statistics}
\label{operation:ShowCropStatistics}
The system shows the statistics of the crop \emph{(Date harvested, Plant
harvested, Number of plants harvested, Total weigth of the harvest)} and
calulates the \emph{Average weigth/plant, Days until the plants were harvested
and the growth per day}.

\begin{description}

\item \textbf{Parameters:} DatePlanted, DateHarvested, PlantHarvested,
NumberOfPlants, TotalWeigthOfCrop
\item \textbf{Precondition:} The system is bootedup and any actor is logged in.
\item \textbf{Post-condition:} The corresponding information will be calculated
and shown.

\item \textbf{Triggering:}
\begin{enumerate}
\item If at any time the \emph{statistics} page is accessed.
\item The logged in user chooses a different crop from the crop list on the
\emph{statistics} page.
\end{enumerate}
\end{description}

\subsection{Example of Show Crop statistics}
The logged in user wants to see the statistics of the past crops.
The user clicks on the \emph{statistics} page in the menu.
The system will now calculate and show the information on the latest crop.
The user wants to see older crops and triggers this system operation by choosing
a different crop.



\section{Automatical Creation of Alert for Sensors}
\label{operation:AddAlertForSensors}
The system checks the incoming signal from all the different sensors for invalid
data or no incoming signal.

\begin{description}

\item \textbf{Parameters:} SensorData
\item \textbf{Precondition:} The system is bootedup.
\item \textbf{Post-condition:} AlertsDatabase is updated with a new entry.

\item \textbf{Triggering:}
\begin{enumerate}
\item The Sensors transmit invalid or impossible data.
\item The Sensors transmit no data.
\end{enumerate}
\end{description}

\subsection{Example of Automatical Creation of Alert for Sensors}
The system is bootedup and is monitoring the signal of the sensors.
Now one of the sensors sends invalid data (example: Humidity Sensor sends data
with more than 100 percent humidity) or lost the connection for whatever reason
to the system, an entry will be added into the alerts database.




\section{Automatical Creation of Alert for Warnings}
\label{operation:AddAlertForWarnigns}
The system checks the incoming signal from the humidity and temperature sensors
for data which is out of the pre-defined range.

\begin{description}

\item \textbf{Parameters:} SensorData
\item \textbf{Precondition:} The system is bootedup.
\item \textbf{Post-condition:} AlertsDatabase is updated with a new entry.

\item \textbf{Triggering:}
\begin{enumerate}
\item The Sensors transmit data out of the pre-defined range
\end{enumerate}
\end{description}

\subsection{Example of Automatical Creation of Alert for Warnings}
The system is bootedup and is monitoring the signal of the sensors.
Now the temperature sensors sends a temperature of 18 degrees celsius. The
pre-defined range for the temperature is from 20 to 27 degrees. The system will
add a warning to the alerts database, the temperature is too low.
