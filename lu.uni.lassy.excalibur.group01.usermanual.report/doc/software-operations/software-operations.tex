\chapter{Software operations}
\label{chap:soptware_operations}


Explain each allowed software operations (i.e. an atomic unit of treatment, a service, a functionality) including a brief description of the operation, required parameters, optional parameters, default options, required steps to trigger the operation, assumptions upon request of the operation and expected results of executing such operation.
Describe how to recognise that the operation has successfully been executed or
abnormally terminated. The template given below (i.e. section \ref{operation:MyOperation} has to be used).

Group the operations devoted to the needs of specific actors. Common
operations to several actors may be grouped and presented once to avoid redundancy.


\section{MyOperation}
\label{operation:MyOperation}
The system operator creates and adds a new crisis to the system after being
informed by a third party (citizen, organization) and selects a crisis handler for the crisis.

\begin{description}

\item \textbf{Parameters:} Reporter Personal Information, Crisis Information, Crisis Handler
\item \textbf{Precondition:} The system operator is logged in and has received information from a reporter.
\item \textbf{Post-condition:} A new crisis has been added to the system and the new crisis has been assigned to a crisis handler, the Handler has received an automatic notification from the system.
\item \textbf{Output messages:} The selected Crisis Handler will be notified
automatically once the crisis has been created.

\item \textbf{Triggering:}
\begin{enumerate}
\item From within the crisis management window fill out the required entries related to the personal information of the reporter such as name and phone number.
\item Fill out the entries related to the crisis type, impacted area, priority, description, GPS coordinates, address and finally choose a Crisis Handler from the combo box.
\item Click on the Submit button in and add the entry to the database.
\end{enumerate}

 
\end{description}

 
\subsection{MyExample1}
Examples should illustrate the use of \textbf{complex operations}.

Each example must show how the actor uses the software operation under
description to achieve (at least one of) its expected outcome.

It might be required to include GUI screenshots to illustrate the example.


\section{Adding task to gardeners schedule}
\label{operation:addTaskGardener}
The manager creates and adds a new task to be added to schedule.


\begin{description}

\item \textbf{Parameters:} Task name, Task description, Room, Gardeners name,
Name of the task
\item \textbf{Precondition:} The manager is logged in
\item \textbf{Post-condition:} A new task has been added to the schedule and the
new task has been assigned to the specified gardener.
\item \textbf{Output messages:} The manager will be notified that the task has
been created.

\item \textbf{Triggering:}
\begin{enumerate}
\item From within the Manager Schedule screen, the manger fills out the 
required entries related to the task information like the name of the task or
the date.
\item Click on the add button and add the task to the schedule.
\end{enumerate}

 
\end{description}

 
\subsection{MyExample1}
The manager wants to add a watering task to the schedule, so fills out the
textinputfields as shown in the image and then he clicks on the add button.



\section{Automatical request for crops}
\label{operation:RequestForCrops}
The system checks the crops if any specifique crop is less then 10 and requests
crops.

\begin{description}

\item \textbf{Parameters:} ListOfCrops
\item \textbf{Precondition:} The system is bootedup and any actor is logged in.
\item \textbf{Post-condition:} Request table from Manager Screen2 updated with
the amount of crops which have to be refilled.

\item \textbf{Triggering:}
\begin{enumerate}
\item System checks every singel amount left of crops in the vegtable table.
\item In case that an amount is less than 10.
\item System send request to the manager.
\end{enumerate}
\end{description}

\subsection{Example of Automatical Frood/Vegtable Request}
Gardener retrives a chosen amount of crops of a specific vegtable or frood.
If the amount of the specifique crop is less than 10.
The System will send the request to the manager request table.


